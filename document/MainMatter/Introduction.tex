\chapter*{Introducción}\label{chapter:introduction}
La formación de competencias profesionales es un aspecto fundamental en el proceso de instrucción en las 
instituciones educativas, principalmente en la educación superior y tecnológica. En la actualidad, 
gracias a los avances tecnológicos, el aprendizaje se ha vuelto más accesible y flexible a través de 
los entornos virtuales de aprendizaje (EVA). Recientemente, la humanidad se encuentra aún dando 
solución a un problema global de pandemia que potenció la educación a distancia y todos sus mecanismos 
de enseñanza, por esta razón la demanda de tecnología digital e innovadora para respaldar las tareas de 
enseñanza, administrar las clases y hacer un seguimiento de los alumnos se ha convertido en una parte 
crucial de la educación.(*insertar referencia) 

En tiempos de pandemia la mayoría de los países fueron forzados a cerrar todas sus instituciones académicas 
y transitar al aprendizaje en línea. Aunque las clases en línea eran portables, de fácil acceso 
y aumentaban las oportunidades de aprendizaje para adultos, todas las escuelas y universidades enfrentaron 
múltiples desafíos al requerir la adopción de programas de enseñanza en línea.(*insertar referencia) Muchos 
docentes y estudiantes además se encontraron con la difícil tarea de tener que continuar con sus cursos 
en este nuevo sistema, teniendo esto un impacto negativo en la calidad del curso. Dentro de los problemas más típicos que 
podemos citar, está el hecho de que los estudiantes se pierden el trabajo de laboratorio y reciben 
menos retroalimentación porque con frecuencia son demasiado tímidos para hacer preguntas durante el curso, 
mientras que los profesores carecen de contacto directo en el aula con los estudiantes, no pueden 
observarlos ni explicar el contenido, y no pueden notificar de inmediato a los estudiantes que pueden 
estar en riesgo. Muchos de estos factores aumentan la probabilidad de que los estudiantes reprueben, 
abandonen y se retiren antes de graduarse. 

Sin embargo, con el aumento de las tecnologías, la tendencia hacia el aprendizaje centrado en el estudiante, 
el cual responde a sus necesidades aumentó sustancialmente (*insertar referencia). Por eso, los docentes y 
las universidades deben proporcionar herramientas de aprendizaje en línea para apoyar la educación 
de manera eficiente. En este sentido (*insertar referencia), Moodle LMS(*insertar significado en pie de pagina) 
ofrece un entorno de aprendizaje con software digital, de libre acceso y ademas de código abierto, que se ha utilizado en muchas universidades de todo 
el mundo, en el cual los estudiantes obtienen acceso rápido y eficiente a los recursos y actividades 
de un curso, mientras los profesores pueden usarlo como una herramienta eficiente para administrar el 
aula a través de cuestionarios, tareas, exámenes y otras actividades. Adicionalmente este te permite recopilar datos 
de la actividad estudiantil y docente, generando un archivo de registros bastante extenso. Esta última ventaja 
se utiliza para el análisis y el pronóstico del rendimiento de un estudiante dentro del curso. Por todas estas 
características este estudio se centrará en cursos impartidos en esta plataforma.

En estos entornos como Moodle es un reto para los profesores obtener una métrica que permita analizar 
y predecir el desempeño de cada estudiante durante un curso en el logro del aprendizaje de cada objetivo 
de estudio. Con esta investigación se busca desarrollar una herramienta que permita proveer a tiempo una 
atención personalizada a los estudiantes, logrando una intervención oportuna que permitiría conducir y 
afianzar la confianza de los mismos dentro del curso, a pesar de las brechas detectadas, en los resultados al 
finalizar este.

Los objetivos de predecir el desempeño que puede tener un estudiante dentro de un curso son:
\begin{itemize}

    \item Entender el aprendizaje académico del estudiante en su trayectoria en cada objetivo 
    académico dentro del entorno: esto además incluye el cómo está progresando hacia la consecución de 
    los objetivos académicos específicos del curso. Esto permite a los profesores y tutores identificar 
    áreas en las que el estudiante puede estar estancado y proporcionar intervenciones específicas para 
    ayudarles a mejorar su rendimiento. Además, también puede ayudar a identificar las fortalezas del 
    estudiante y permitir que se les brinden oportunidades para desarrollarlas aún más.

    \item Facilitar a los profesores entender las condiciones y realidades académicas 
    en las que están los estudiantes, esto incluye factores como la carga de trabajo del estudiante, 
    su nivel de motivación, su capacidad de concentración y otros factores que pueden influir en su 
    rendimiento académico. Al comprender mejor estas condiciones y realidades, los profesores 
    pueden ajustar su enseñanza y proporcionar apoyo personalizado para ayudar a los estudiantes 
    a alcanzar su máximo potencial

    \item Implementar planes personalizados de recomendación. Una vez que se ha realizado la predicción 
    del rendimiento del estudiante, se pueden implementar planes personalizados de recomendación para 
    ayudar al estudiante a mejorar su rendimiento. Estos planes pueden incluir guías específicas 
    para el estudio, la práctica y la preparación para exámenes, así como para el apoyo académico adicional, 
    como tutorías o asesoramiento individualizado.

\end{itemize}

Una forma de detectar a los estudiantes que pudieran tener un desempeño medio o bajo dentro de un curso, 
es realizar predicciones tempranas de sus calificaciones en los diferentes temas que se abordan.

El uso de modelos de predicción para el análisis del aprendizaje, así como los sistemas de recomendación, 
presentan múltiples desafíos, especialmente el cómo obtener, procesar y usar los datos para construir el modelo de 
predicción apropiado. A este campo dentro de la analítica del aprendizaje se le llama: Minería de datos educacionales. 
En este estudio se proponen los siguientes indicadores, los cuales son extraídos del archivo de registros de Moodle: 
\begin{itemize}
    \item Cantidad de veces que el estudiante ha ingresado a las actividades y recursos de la plataforma.
    \item Nivel de participación en los tipos diferentes de actividades.
    \item Resultado de la evaluación que se otorga por el docente en cada actividad interactiva de cada objetivo académico.
\end{itemize}

Este preprocesamiento de los datos a realizarse en esta investigación tiene significativa importancia puesto que 
interviene directamente en la precisión y en el costo computacional del modelo predictivo a emplear.

Luego, ¿Sería posible establecer un modelo de predicción y un sistema de recomendación en los 
Entornos Virtuales de Aprendizaje que facilite el proceso de enseñanza con el aprendizaje personalizado de los 
estudiantes en los cursos montados en la plataforma de una institución? 
En el contexto de la presente tesis, el objetivo general es responder a la interrogante anterior desarrollando 
un modelo predictivo y un sistema de recomendación a partir de los datos extraídos y preprocesados de Moodle. 

Como resultado de este trabajo se pretende predecir el comportamiento de un estudiante en un 
curso y, a partir de ella, se busca ofrecer recomendaciones personalizadas adaptadas a las 
necesidades y preferencias individuales de cada estudiante, con el fin de mejorar su 
desempeño en el proceso de aprendizaje y aumentar su rendimiento académico.

Para lograr este objetivo se tendrán en cuenta los siguientes pasos:
\begin{enumerate}
    \item Recopilación y limpieza los datos pertinentes.
    \item Construir el modelo predictivo: Implementación de algoritmo de clasificación de aprendizaje de máquina.
    \item Construir un sistema de recomendación, basado en el historial del estudiante en el entorno virtual y 
    los requerimientos del curso donde se encuentra, a partir de la predicción obtenida.
\end{enumerate}
\addcontentsline{toc}{chapter}{Introducción}