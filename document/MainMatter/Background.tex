\chapter{Marco teórico-conceptual}\label{chapter:state-of-the-art}

\section{Evolución de los LMS}
La sociedad contemporánea tiene la transmisión de conocimientos como algo determinante para garantizar el futuro, históricamente la educación se ha tornado un elemento fundamental para el desarrollo de la sociedad por lo que se estableció como una actividad necesaria en la vida de todas las personas.

%(https://bit4learn.com/es/lms/)
A través del tiempo la transmisión de información ha evolucionado de la mano de las tecnologías disponibles en cada época, en este apartado se expone de una de las herramientas de transmisión de la información más utilizadas en la educación en nuestra era, los Sistemas de Gestión de Aprendizaje, conocidos también como LMS, por sus siglas en inglés Learning Management. [\cite{bit4learn}]

\subsection{¿Qué es un LMS?}

Un Sistema de Gestión de Aprendizaje (\textit{Learning Management System}, LMS) es un tipo de software o tecnología soportada en línea que permite crear, implementar y desarrollar un programa de entrenamiento o un proceso de aprendizaje específico. Estos sistemas son utilizados por organizaciones empresariales, agencias gubernamentales; así como, instituciones educativas tradicionales, en la búsqueda de complementar o mejorar métodos educativos tradicionales mientras ahorran tiempo y recursos. Como principales características de estos entornos se puede citar: 
\begin{itemize}
    \item \textbf{Diseño responsivo}: Los usuarios podrán acceder al sistema desde cualquier dispositivo que elijan, ya sea computadora de escritorio, teléfono inteligente, tableta o laptop. El LMS desplegará automáticamente la versión más adecuada para cada uno de los dispositivos. Adicionalmente, los usuarios podrán descargar contenido para trabajar sin tener que estar conectados a internet.
    \item \textbf{Interfaz amigable}: La interfaz del usuario permitirá a los estudiantes navegar fácilmente por la plataforma y mantendrá alineados los objetivos y habilidades, tanto del usuario como de la institución.
    \item \textbf{Reportes y analíticas}: Debe integrar herramientas de evaluación. Los profesores y administradores deben de contar con la capacidad para visualizar y revisar sus iniciativas didácticas para determinar su efectividad y realizar ajustes en caso de que sea necesario.
    \item \textbf{Manejo de cursos y catálogos}: Debe contener todos los cursos disponibles y accesibles para que administradores y profesores puedan manejar el catálogo y dirigirlo hacia una experiencia de aprendizaje enfocada a resultados.
    \item \textbf{Contenido interoperable e integrado}: El contenido creado y almacenado en un LMS debe empaquetarse de acuerdo a estándares educativos regulados por instituciones\footnote[1]{SCORM : \textit{Shareable Content Object Reference Model}, es uno de estos estándares. Estos empaquetados permitan crear objetos pedagógicos estructurados con objetivos fundamentales de facilitar la portabilidad de contenido de aprendizaje, poder compartir y reusarlo}.
    \item \textbf{Servicio de soporte}: Los diferentes desarrolladores deben ofrecer distintos niveles de soporte que brinda la plataforma como foros de discusión, donde los usuarios puedan interactuar entre ellos, los \textit{chats}, los números telefónicos de ayuda, etc.
\end{itemize}

\subsection{Resumen histórico}
%https://es.wikipedia.org/wiki/Programmed_Logic_Automated_Teaching_Operations
El surgimiento de los software educativos parte desde 1960 con Plato (\textit{Programmed Logic for Automated Teaching Operations}, Lógica programada para operaciones de enseñanza automatizadas) [\cite{Wiki_Plato}], considerado el primer software educativo, fue desarrollado en la Universidad de Ilinois por Donald BItzer. El sistema contaba con una interfaz táctil y los programas educativos eran desarrollados en Tutor, un lenguaje de programación orientado a la educación que contaba con herramientas como sistemas de grabación y ficheros de notas.  
%http://www.virtualeducation.wiki/index.php/NKI_Nettstudier

En 1987 NKI, universidad de Noruega desarrolla el primer sistema de educación a distancia, NKI Distance Education [\cite{NKI}], con funcionalidades básicas de almacenamiento de recursos en línea, accesibles por todos los usuarios pertenecientes a la plataforma.  
%(https://www.comparasoftware.com/firstclass-lms)

En 1990 nace FirstClass \cite{FirstClass} un producto de Softarc una empresa tecnológica Sueca. al principio esta plataforma estuvo enfocada a solo los sistemas operativos de Macintosh(macOS). Esta plataforma incorporaba funcionalidades como foros y correos. Se puede decir que Firsclass fue la primera plataforma de \textit{e-learning} que propició la interconexión entre profesores y alumnos.  
%https://docs.moodle.org/all/es/Acerca_de_Moodle

\textit{Modular Object-Oriented Dynamic Learning}, Moodle por sus siglas en inglés [\cite{Moodle_docs}]. Probablemente la plataforma más conocida en la historia de los LMS fue lanzada en 2002 por Martin Dougiamas. En cuanto a las principales diferencias de esta plataforma respecto a sus antecesoras es la posibilidad de asignar roles como administrador, profesor y estudiante. Al ser una tecnología de código abierto, Moodle se ha convertido en una de los entornos preferidos en el momento de implementar proyectos académicos en entidades docentes, por su fácil adaptabilidad y personalización en base a las necesidades de cada institución. El presente estudio se centra en este LMS, en el siguiente apartado donde se profundiza acerca de sus principales características, funcionalidades y beneficios.  


En 2008 los LMS comienzan a brindar sus servicios desde la nube a través del modelo SAAS(Software as a Service). La principal ventaja de brindar el servicio directamente desde la nube, esto trajo un avance muy grande para el mundo de las LMS ya que facilitaba el acceso a la información educativa en cualquier lugar y desde cualquier dispositivo sin necesidad de instalaciones tediosas. El modelo SAAS se volvió el modelo más aplicado en cuanto a servicios de LMS, actualmente más del 70\% de los LMS actuales ofrecen sus servicios bajo este modelo.  


La actualidad de los LMS están cubiertos por tendencias que marcan el nuevo camino de estas plataformas, entre ellas se encuentran:
\begin{itemize}
    \item \textbf{Aprendizaje Social}: El aprendizaje social o también conocido como \textit{social learning} es una de las tendencias que las últimas plataformas LMS están incorporando, plataformas como Google Classroom o Edmodo llevan este tipo de tendencias en donde la información se presenta en un canal de comunicación y la información más antigua va quedando al final.
    \item \textbf{Gamificación}: La gamificacion es la aplicación de dinámicas propias de los juegos a los procesos de aprendizaje, aspectos como los puntos o los premios forman parte de esta herramienta, muchas plataformas han comenzado a incorporar estas funcionalidades para poder mejorar la interacción y el compromiso de los estudiantes.
    \item \textbf{Experiencia de Usuario}: La experiencia de usuario se ha tornado un factor determinante para la creación de experiencias de aprendizajes, el desarrollo de interfaces y el estudio del comportamiento del usuario para refinar los detalles de la plataforma dan una ventaja para la presentación de contenidos educativos. Los videos se han vuelto los más populares en cuanto a transmisión de contenidos, tener una plataformas simple y sencilla se ha vuelto en uno de los requerimientos más básicos para los proyectos de aprendizaje en línea.
\end{itemize}

\subsection{Futuro de los LMS}

El futuro de los LMS, como software en general, estará marcado por la flexibilidad y la adaptabilidad que pueda ofrecer a las nuevas tecnologías que surjan. En este sentido es importante destacar que la inteligencia artificial y los modelos de aprendizaje automático tendrán un papel fundamental en dicho futuro, herramientas que permitan analizar a profundidad los datos históricos almacenados en un LMS con el objetivo de entender la línea de aprendizaje de cada estudiante y de esa manera poder personalizar al máximo lo que se muestra a cada usuario, marcarán sin duda el camino de los LMS en unos años. En ese sentido, la presente tesis aborda, posibles soluciones al análisis de esos datos y cómo emplearlos para el beneficio del usuario.


\section{Modular Object-Oriented Dynamic Learning: MOODLE}
%(https://stats.moodle.org)
%(https://moodle.org/stats/)
Moodle [\cite{Moodle_Solutions}] es una plataforma de aprendizaje diseñada para proporcionarle a educadores, administradores y estudiantes un sistema integrado único, robusto y seguro para crear ambientes de aprendizaje personalizados. Soportada financieramente por una red mundial de más de 80 compañías de servicio, impulsando a cientos de miles de ambientes de aprendizaje globalmente, cuenta con la confianza de instituciones y organizaciones grandes y pequeñas. El número de usuarios a nivel mundial, ascendió a más de 200 millones de usuarios en agosto del 2020 \cite{Moodle_stats}, entre usuarios académicos y empresariales, lo cual la convierten en la plataforma de aprendizaje más utilizada del mundo.

\subsection{Principales características}
Este estudio se centrará en la versión estable de Moodle 4.2, que al igual que las anteriores posee una variedad considerable de características, entre las que se puede destacar:  
\begin{itemize}
    \item \textbf{Interfaz moderna y fácil de usar}: Diseñada para ser responsiva y accesible, la interfaz de Moodle es fácil de navegar, tanto en computadoras de escritorio como en dispositivos móviles.
    \item \textbf{Actividades y Herramientas colaborativas}: Cuenta con actividades interactivas de uso muy intuitivo para el usuario; los foros, las wikis, los glosarios, actividades de bases de datos son ejemplos de herramientas que se pueden desarrollar en Moodle con el objetivo de potenciar el aprendizaje colaborativo y hacer más ameno el proceso de aprendizaje.
    \item \textbf{Gestión conveniente de archivos}: Cuenta con la opción de trasladar archivos desde servicios de almacenamiento en la nube, incluyendo MS OneDrive, Dropbox y Google Drive.
    \item \textbf{Diseño personalizable de la plataforma}: Los administradores tienen la posibilidad de personalizar y adaptar la interfaz de Moodle a las necesidades de su institución u organización. 
\end{itemize}

La plataforma es de código abierto y gratuita, cualquier persona puede adaptar, extender o modificar Moodle, tanto para proyectos comerciales como no-comerciales, sin pago de cuotas por licenciamiento, esto, además, significa que Moodle siempre está en proceso de mejora y actualización, cuenta con una comunidad fiel de desarrolladores y con estándares de código para su mejora. Los administradores de las instituciones se pueden apoyar en esto para flexibilizar aún más la plataforma y de esa forma cubrir, de una manera más eficiente, las necesidades de dichas instituciones.  


Su configuración modular y diseño inter-operable les permite a los desarrolladores el crear \textit{plugins}[\cite{Moodle_Plugins}] e integrar aplicaciones externas para lograr funcionalidades específicas, además, cuenta con una base de datos relacional (SQL) perfectamente escalable para soportar, desde clases pequeñas con pocos estudiantes hasta grandes organizaciones con millones de usuarios.  


Para la comunicación con esta base de datos desde la perspectiva de un programador, Moodle cuenta con un API (del inglés, \textit{Application Programming Interface}, en español, Interfaz de Programación de Aplicaciones), con todas las funciones necesarias para su modificación desde el código, permitiendo de esta manera extender MOODLE a aplicaciones externas o herramientas que monitoreen la plataforma en cuestión.  


En esta investigación se utiliza esta API como principal forma de extracción de datos de la plataforma, en el siguiente capítulo se profundiza acerca de las diferentes funciones utilizadas así cómo la correcta forma de usarlas.  


Otra de las características a tener en cuenta es que Moodle almacena, en su extensa base de datos, cada acción realizada por el usuario con fecha y hora, administradores y profesores tienen acceso a este registro de actividad realizada por los usuarios en sus respectivos cursos, o en la plataforma viéndolo de una manera más general, Moodle posibilita esto a través de reportes, accesibles en la misma plataforma y con posibilidad de descarga para su posterior análisis. A estas acciones se les llama "\textit{logs}", el problema principal de estos reporte es que son muy tediosos de analizar por una persona, por lo que varios estudios y herramientas se han centrado en el análisis de estos, para de alguna manera "traducirlos" a un lenguaje más entendible por la persona interesada.

En este sentido Brian Sal Sarria, de la Universidad de Cantabria en España, en su tesis de grado de Ingeniería Informática [\cite{Sarria_tesis}] plantea una guía para la traducción de estos \textit{logs}, en donde para acceder a estos, se debe descargar el reporte mencionado anteriormente. Por otra parte en el estudio realizado por Cristóbal Romero Morales, del departamento de Ciencias de la Computación y Análisis Numérico de la Universidad de Córdoba en España, titulado " Aplicando Minería de Datos en Moodle " [\cite{Romero_tesis}], utiliza un enfoque de trabajo directo con la base de datos de Moodle, emplea el lenguaje SQL para acceder a la información de las tablas de dicha base de datos. En el capítulo 2 se plantea la propuesta llevada a cabo en la presente tesis.  


%(https://moodledev.io/docs/apis/plugintypes/mlbackend)
\subsection{Futuro}
El futuro de esta plataforma está altamente influenciado por la comunidad de desarrolladores que posee, así como por los usuarios que lo utilizan. Cualquier idea novedosa es bienvenida siempre que cumpla con los estándares de la plataforma. Como mismo el futuro de los LMS en general está marcado por la inteligencia artificial, Moodle no es una excepción, ya se trabaja en modelos de aprendizaje automático que se integrarán a la plataforma, con el objetivo de potenciar el aprendizaje personalizado y hacer de esta un lugar mucho más amigable, eficiente y productivo para estudiantes y profesores, también se valora la posibilidad de que cualquier desarrollador que proponga un modelo o algoritmo para resolver una problemática determinada, tenga su espacio en la plataforma, ya sea integrando el modelo o como una herramienta externa en forma de \textit{plugin}.  


\section{Minería de datos educacionales}

Las instituciones docentes disponen de una gran cantidad de información, la cual posee un alto valor pedagógico y una gran importancia para la evolución del proceso de aprendizaje. Este tipo de información puede utilizarse en la toma de decisiones para mejorar sus estrategias y políticas docentes educativas. Además, las estadísticas y el flujo de información que se produce durante el análisis de los datos educativos, favorecen al sector docente para profundizar en sus investigaciones. No obstante, en la mayoría de las instituciones educativas, en sus entornos virtuales de aprendizaje, el gran volumen de información disponible no es aprovechado al máximo.  


Para lidiar con esta situación, se han llevado a cabo múltiples esfuerzos en el diseño de herramientas que utilizan técnicas de extracción de conocimiento para procesar los datos obtenidos en los entornos educativos [\cite{Sebastian}]. La disciplina que engloba este grupo de técnicas y metodologías se conoce como Minería de Datos Educacionales (educational data mining - EDM) y recientemente se ha evidenciado un significativo avance en esta línea de investigación [\cite{Vasile,Dominik}].  


Los objetivos que se persiguen al aplicar esas técnicas dependen de a quién va dirigido el conocimiento extraído. Los distintos roles que utilizan en EDM se pueden clasificar en tres categorías principales, estudiantes, profesores e instituciones académicas [\cite{Sebastian}].  


Desde el punto de vista del estudiante tiene como objetivos establecer qué actividades, recursos y tareas podrían mejorar su rendimiento académico y su motivación. También resulta importante determinar qué actividades se ajustan mejor a su perfil y fijar qué camino recorrer para obtener un resultado concreto, basado en el desempeño del alumno en el camino recorrido hasta el momento, así como, por comparación con lo realizado por otros estudiantes
de características similares [\cite{Sebastian}].  


Desde el punto de vista del profesor, con la EDM se pueden resolver múltiples problemas y tareas de gran incidencia en la efectividad de los métodos de aprendizaje [\cite{Sebastian}].  


Uno de los objetivos primarios seria evaluar la eficacia del proceso de enseñanza y en función de esta corregir el contenido de los cursos para mejorar la estructura de los mismos. Para llevar a cabo esta tarea, resulta indispensable monitorizar cada actividad y determinar su grado de dificultad y los errores más frecuentes cometidos en su ejecución.  


Otro aspecto de gran importancia para los docentes es identificar las relaciones existentes entre los usuarios para organizarlos en grupos homogéneos. Este tipo de organización permitiría encausar los intereses comunes de los estudiantes para fomentar el desarrollo de grupos participativos y líneas de investigación, colaborando con personas dedicadas a temáticas afines.  


Además, se puede incrementar la motivación de los estudiantes al personalizar y adaptar el contenido de cursos diseñando los planes de instrucción a partir de las características de cada grupo. Otra tarea consiste en investigar el comportamiento de los alumnos y su desempeño en los cursos para buscar patrones generales y patrones anómalos en su rendimiento.  


Finalmente desde el punto de vista de las instituciones académicas los objetivos de la EDM son mejorar la eficiencia de los sitios web educativos así como adaptarlo a los hábitos de sus usuarios teniendo en cuenta el tamaño de servidor óptimo y la distribución del tráfico en la red [\cite{Sebastian}].  


La minería de datos educacionales se ha convertido en una herramienta imprescindible en el entorno de aprendizaje virtual, permitiendo a educadores y administrativos mejorar la efectividad del proceso de enseñanza y aprendizaje. Este campo interdisciplinario se aprovecha de los métodos de la minería de datos para analizar datos procedentes de contextos educativos con el fin de seguir el comportamiento de los usuarios en el entorno y a partir de ahí, resolver los problemas pertinentes.  


El proceso de minería de datos en entornos de aprendizaje virtual, como Moodle, involucra varios pasos fundamentales. Moodle, una plataforma de aprendizaje gestionada, proporciona un amplio rango de datos que pueden ser utilizados para este fin, incluyendo información sobre la interacción de los estudiantes con los materiales del curso, sus patrones de navegación, resultados de evaluaciones, participaciones en foros y muchos otros indicadores.  


\textbf{Pasos en la Minería de Datos Educativos en Moodle}:

\begin{enumerate}
    \item \textbf{Recopilación de Datos}: En Moodle, la recopilación de datos puede realizarse manualmente o de forma automatizada.  
        \begin{itemize}
            \item \textbf{Manualmente}: El administrador o el profesor puede descargar registros de actividades, calificaciones, y contribuciones en foros directamente desde la interfaz de Moodle en formatos como .csv o .xls. Estos registros contienen todo lo que ha hecho un usuario en la plataforma, contiene campos como: Componente Afectada, Nombre del evento, Descripción, Fecha y Hora, que son fundamentales para el análisis del comportamiento del usuario [\cite{Sarria_tesis}]. En este estudio se utilizan todos los campos anteriormente mencionados.  
            \item \textbf{Automatizada}: Se puede utilizar API's de Moodle o \textit{plugins} específicos que recolectan datos continuamente y los almacenan en una base de datos externa más robusta para su posterior análisis, además que se pueden realizar peticiones a la base de datos para extraer todos estos registros, que tienen los mismos campos mencionados anteriormente.[\cite{Romero_tesis}]
        \end{itemize}
    En la presente tesis se utilizó un enfoque combinado, manejando el excel extraído manualmente y la obtención de datos necesarios a través de la API que brinda Moodle, todo esto se verá en el siguiente capítulo.  

    \item \textbf{Preprocesamiento}: Los datos extraídos deben ser limpiados y transformados para ser utilizados eficientemente. Esto puede implicar la eliminación de \textit{outliers}, la corrección de errores y la transformación de datos en formatos que los algoritmos de minería de datos puedan procesar. En este sentido el campo de Descripción que viene en esos datos extraídos juega un papel esencial ya que tiene la mayor información a extraer, vienen datos como el id del usuario en cuestión, el id del curso afectado, así como el módulo utilizado. Como este campo es un texto la mejor técnica a emplear son las expresiones regulares,[\cite{Sarria_tesis}] de esta forma se pueden obtener los datos necesarios para el posterior análisis. En este paso también es importante escoger las características de esos datos que se van a emplear.  
    
    
    En la bibliografía revisada hay mucha variedad en cuanto a los campos que se van a utilizar, además esto tiene mucho que ver con la problemática a resolver. En el caso de la predicción académica es común el uso de datos cuantitativos sobre la utilización de un curso por un estudiante, esto incluye por ejemplo: la cantidad de veces que el estudiante accedió a un recurso, la cantidad de veces que un estudiante participó en un forum, la cantidad de intentos realizados en un cuestionario.[\cite{Rodolfo}] Otros estudios además incluyen factores demográficos como el sexo, el lugar de residencia, el horario en el que acceden a la plataforma.[\cite{Sushil}] En este paso se debe ser capaz de transformar los datos extraídos en un formato que cumpla con esas características por cada estudiante.  

    \item \textbf{Análisis de Datos}: Utilizando algoritmos de minería de datos, se busca identificar patrones y tendencias en los datos. Esto implica técnicas como la agrupación, clasificación, reglas de asociación y análisis de secuencias. Dentro de este aspecto hay varios enfoques para el análisis en dependencia de la problemática en cuestión. Las principales técnicas de minería de datos que se emplean en este campo son:
    \begin{enumerate}
        \item \textbf{Agrupamiento(Clustering)}: Se utiliza principalmente para identificar grupos de estudiantes con patrones de comportamiento similares.[\cite{Romero_tesis}]
        \item \textbf{Reglas de asociación}: dentro de estas reglas el algoritmo Apriori ayuda a encontrar qué recursos o actividades están comúnmente están asociados con altas calificaciones.[\cite{Romero_tesis}]
        \item \textbf{Algoritmos de clasificación}: dentro de estos algoritmos se encuentran, los árboles de decisión, el \textit{Random Forest}, regresión lineal, regresión logística, \textit{Support Vector Machine}, redes neuronales, etc. Los cuales a partir de parámetros o atributos y una o varias clases de salida dentro de un conjunto de datos, realiza la clasificación.  

        En este estudio se utilizaron los algoritmos de clasificación para resolver la problemática de predecir el desempeño de un estudiante dentro de un curso virtual. Específicamente los algoritmos: Arboles de decisión, \textit{Random Forest}, regresión lineal, regresión logística y \textit{Support Vector Machine}.
    \end{enumerate}
		
    \item \textbf{Resultados}: Los resultados deben ser interpretados y presentados de tal manera que los educadores puedan entenderlos fácilmente y tomar decisiones informadas para mejorar las estrategias de enseñanza. Esta etapa se divide a su vez en 3 partes:
    \begin{enumerate}
        \item \textbf{Visualización de Datos}: Representación gráfica de los resultados para facilitar su interpretación.
        \item \textbf{Evaluación estadística}: Deben realizarse pruebas estadísticas para determinar la importancia de los patrones encontrados.
        \item \textbf{Validación}: Comprobar los resultados obtenidos con expertos o verificar que se siguen cumpliendo estos patrones en cursos venideros.
    \end{enumerate} 

\end{enumerate}
Esta metodología de trabajo dividida en estos 4 pasos es la seguida por este estudio la cual se abundará en el siguiente capítulo. En el siguiente epígrafe se verá las diferentes formas en que se puede aplicar el aprendizaje automático para resolver el problema de predicción académica.  

\section{Modelos de Aprendizaje Automático}

El aprendizaje automático o aprendizaje de máquinas (del inglés, ML: \textit{Machine Learning}) es una rama de la Inteligencia Artificial, cuyo objetivo es desarrollar técnicas que permitan a los sistemas computacionales aprender mediante un modelo basado en los datos históricos de una entidad, que facilite el análisis y deducción del comportamiento de indicadores de la gestión, en aras de mejoras en la mencionada entidad.  


El aprendizaje automático también está estrechamente relacionado con el reconocimiento de patrones puede ser visto como un intento de automatizar métodos matemáticos que faciliten el proceso de inducción del conocimiento.  


El aprendizaje automático tiene una amplia gama de aplicaciones, incluyendo motores de búsqueda, diagnósticos médicos, detección de fraude en el uso de tarjetas de crédito, análisis de mercado para los diferentes sectores de actividad, clasificación de secuencias de ADN, reconocimiento del habla y del lenguaje escrito, juegos y robótica, evidentemente puede ser utilizado para el comportamiento del desempeño de los estudiantes en un determinado curso académico.  


El aprendizaje automático tiene como resultado un modelo para resolver determinado problema. Entre los modelos se distinguen: los modelos geométricos, los probabilísticos, los lógicos, de agrupamiento y de clasificación.  


Los algoritmos de aprendizaje automático conforman un conjunto de datos para crear un modelo. A medida que se introducen nuevos datos de entrada en el algoritmo de aprendizaje automático, se utiliza el modelo desarrollado para realizar una predicción.  


Los pasos clave para crear un modelo de aprendizaje automático son:  
\begin{enumerate}
    \item Recopilación de datos: Compilación de información confiable para informar al modelo predictivo.
    \item Preparación de datos: Realizar la preparación de los datos, como agrupar y seleccionar los datos relevantes. Los datos se dividen en dos conjuntos: los datos de entrenamiento que se utilizan en el aprendizaje y los datos de evaluación que sirven para medir la efectividad del modelo una vez entenado.
    \item Elegir un modelo: Existen muchos prototipos de aprendizaje automático, y algunos se adaptan mejor a casos de uso específicos que otros, por ende se debe seleccionar aquel que garantice mejorar la eficacia y precisión con el tiempo.
    \item Entrenamiento: Los datos refinados son elegidos para mejorar la capacidad predictiva del modelo.
    \item Evaluación: Introducir de nuevos datos para comprobar la efectividad de sus capacidades predictivas.
    \item Ajuste de parámetros: Ajustar los parámetros de prueba específicos que puedan amoldarse para producir mejores resultados.
\end{enumerate}
    
Existen varios tipos de algoritmos de aprendizajes: 
\begin{itemize}
    \item Aprendizaje supervisado: Los algoritmos de aprendizaje supervisado basan su aprendizaje en un juego de datos de entrenamiento previamente etiquetados.
    \item Aprendizaje no supervisado: Los métodos no supervisados (\textit{unsupervised methods}) son algoritmos que basan su proceso de entrenamiento en un juego de datos sin etiquetas o clases previamente definidas. Está dedicado a las tareas de agrupamiento, también llamadas \textit{clustering} o segmentación, donde su objetivo es encontrar grupos similares en el conjunto de datos.
    \item Aprendizaje semi-supervisado: es una clase de técnicas de aprendizaje automático que utiliza datos de entrenamiento tanto etiquetados como no etiquetados: normalmente una pequeña cantidad de datos etiquetados junto a una gran cantidad de datos no etiquetados.
    \item Aprendizaje por refuerzo: es un área del aprendizaje automático inspirada en la psicología conductista, cuya ocupación es determinar qué acciones debe escoger un agente de software en un entorno dado con el fin de maximizar alguna noción de "recompensa".
\end{itemize}

Para este estudio se utilizaron algoritmos de aprendizaje supervisado (Árboles de decisión, \textit{Random Forest}, regresión lineal, regresión logística, \textit{Support Vector Machine}), ya que los datos recopilados están previamente etiquetados.  


El aprendizaje automático se ha utilizado en el proceso de enseñanza-aprendizaje haciéndose énfasis en los entornos virtuales de aprendizaje convirtiéndose en una dirección importante de las aplicaciones de minería de datos educativos [\cite{murad2018recommendation}], en aras de lograr mejoras en la identificación del comportamiento de los estudiantes, la evaluación de calificaciones de aprobado, reprobado y, especialmente en la predicción del desempeño de los estudiantes. Por tanto, el desarrollo y combinación de educación y aprendizaje automático es la tendencia actual en los centros educacionales [\cite{Suleiman}].  



La mayoría de los trabajos publicados se concentran en el uso de un único algoritmo de clasificación. Por ejemplo, en [\cite{conijn2016predicting}] se utilizó la regresión logística para analizar cursos de variables predictivas  de LMS y predecir las calificaciones de los estudiantes durante 10 semanas. Los resultados indicaron que las calificaciones de las evaluaciones se relacionan con las calificaciones finales, mientras que los eventos como debates, foros o uso de wiki son un predictor menos confiable de la calificación final.


En [\cite{ljubobratovic2019using}], se aplicó \textit{Random Forest} para construir un modelo utilizando eventos de registro (conferencias, cuestionarios, laboratorios, videos) para predecir el fracaso de los estudiantes dentro de un curso, la precisión fue de 96.3\%. Revelaron que los resultados de las puntuaciones de laboratorio son el predictor más sólido en este estudio.

Asimismo, [\cite{nguyen2021using}] utiliza datos de cuatro cursos, como material semanal, videos conferencias, cuestionarios y ejercicios, de la Universidad Van Lang en Vietnam, aplicando un clasificador de regresión lineal para pronosticar el riesgo de reprobar el curso. Se encontró que los estudiantes con menos del 37\% de interacción estaban en riesgo de reprobar. Se realizó un análisis de 30000 expedientes estudiantiles [\cite{maraza-quispe2021predictive}] que incluyó cinco indicadores como rendimiento académico, tareas, acceso, aspectos sociales y cuestionarios, los cuales revelaron que luego de aplicar el árbol de regresión, el modelo implementado tuvo una tasa de precisión de 89.70\%.  


En [\cite{mi2022research}] se propone un árbol de decisión para predecir los resultados de los estudiantes en riesgo en tres niveles (alto, promedio y bajo). Se recopilaron muchas características (género, sesión, duración de la clase, GPA, especialización título, año, asistencia y puntuación de mitad de período) de los estudiantes del curso de Introducción a la Programación en el Buraimi University College en Omán. La precisión del modelo fue del 87.88\%, demostrando su efectividad.  


De manera similar se propuso el modelo de árbol de decisión para predecir las tasas de abandono estudiantil en la Universidad Phayao en Tailandia [\cite{khan2021machine}]. Se analizó un conjunto de datos de 397 estudiantes y se consideraron las causas de deserción. El resultado produjo una precisión general de alrededor del 87.21\% de precisión.  


Aunque el rendimiento de la predicción funciona con un único algoritmo, en este estudio se exploran y comparan diferentes modelos con el objetivo de encontrar el de mejor precisión. Además, ninguno ofrece una comparación de la predicción en las diferentes etapas del curso, y tampoco se utilizan diversos predictores de entrada en términos de predicción temprana del desempeño de los estudiantes. En el siguiente capítulo se abordará con más profundidad este tema.


