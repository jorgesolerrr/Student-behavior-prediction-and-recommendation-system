\begin{conclusions}
    Conclusiones  

    Como resultado del presente trabajo se logró la creación de una solución computacional que responde a la problemática de predicción académica, en la que se integran 
    técnicas estadísticas, de análisis de datos y de aprendizaje automático. La solución se caracteriza por la construcción de modelos de datos educacionales y de aprendizaje automático con vista a favorecer el análisis descriptivo del comportamiento de un estudiante dentro de un curso virtual, así como 
    el uso de las potencialidades que brinda una LMS como Moodle.  

    A partir de la profundización en las áreas de conocimiento asociadas, el acercamiento al contexto educativo y 
    el estudio de soluciones similares a la predicción del desempeño académico. Se logró el diseñó de una metodología general para el procesamiento de datos educacionales de la plataforma Moodle, a partir de la cual, se implementó un prototipo sobre cuya base se aplicó un conjunto de experimentos que 
    permitió establecer la validez de la concepción global.  

    La creación de una solución computacional basada en los avances científicos y tecnológicos para el análisis de los datos históricos de los estudiantes, la construcción de diferentes conjuntos de datos con distintos atributos implicados, la selección de características dentro de estos, el procesamiento del lenguaje natural a la hora de limpiar los datos, así como realizar la predicción en las diferentes etapas del curso 
    constituyen aportes con respecto a las herramientas o soluciones computacionales implementadas con anterioridad en el ámbito de la analítica del aprendizaje.  
    
    Los resultados de la investigación permiten responder afirmativamente a la pregunta científica, ya que ha sido posible la predicción del desempeño estudiantil a partir de modelos de aprendizaje automático, incorporando nuevas formas del procesamiento de los datos de Moodle. 

    
\end{conclusions}
