\begin{resumen}
	Durante la pandemia de COVID-19, muchas universidades estuvieron en su mayor parte cerradas y sus aulas se transformaron 
	a un formato totalmente en línea. Para los profesores era un desafío gestionar el aprendizaje virtual y, especialmente, realizar 
	un seguimiento del comportamiento de los estudiantes, ya que no se podía establecer contacto interpersonal y por ende, el desempeño de los estudiantes no es fácil de controlar. Para aliviar 
	este problema, una solución, que se ha vuelto cada vez más importante, es la predicción del desempeño de los estudiantes en función 
	de sus datos históricos de registro en los entornos virtuales de aprendizaje. Este estudio, por tanto, tiene como objetivo analizar datos de comportamiento de los alumnos aplicando técnicas de aprendizaje automático a los registros de Moodle, con un total de $453941$ registros. Se utilizaron cinco algoritmos de aprendizaje automático 
	(\textit{Random Forest}, Árbol de Decisión, Regresión Logística, Regresión Lineal y \textit{Support Vector Machine}) para realizar la predicción académica. Además, se crearon dos conjuntos de datos con atributos distintos, los cuales se dividieron en cuatro etapas de progreso del curso 
	(25\%, 50\%, 75\% y 100\%), a los que se le aplicó un proceso de selección de características con el algoritmo Boruta.  

	Los modelos de predicción podría guiar estudios futuros, motivar la autopreparación y reducir las tasas de abandono. 
	En la investigación se evaluaron los modelos con validación cruzada quíntuple. Los resultados indicaron que en ambos conjuntos de datos el algoritmo de Regresión Lineal tuvo el mejor comportamiento en todas las etapas del curso.  

	Los resultados podrían aplicarse a otros cursos y en un registro más grande en entornos virtuales de aprendizaje que tengan condiciones similares de actividad de los estudiantes, en búsqueda de lograr una predicción más precisa del desempeño de los estudiantes.
\end{resumen}

\begin{abstract}
	During the COVID-19 pandemic, many universities were mostly closed, and their classrooms shifted to a fully online format. For teachers, managing virtual learning posed a challenge, especially in monitoring student behavior, as interpersonal contact was not possible, making it difficult to control student performance. To alleviate this problem, an increasingly important solution has been predicting student performance based on their historical log data in virtual learning environments.  

This study aims to analyze student behavior data by applying machine learning techniques to Moodle logs, totaling 453,941 records. Five machine learning algorithms (Random Forest, Decision Tree, Logistic Regression, Linear Regression, and Support Vector Machine) were used for academic prediction. Additionally, two datasets with different attributes were created, divided into four course progress stages (25\%, 50\%, 75\%, and 100\%), subjected to feature selection using the Boruta algorithm.  


The predictive models could guide future studies, motivate self-preparation, and reduce dropout rates. The research evaluated the models with five-fold cross-validation. The results indicated that in both datasets, the Linear Regression algorithm performed the best at all course stages.  


These findings could be applied to other courses and in a larger dataset in virtual learning environments with similar student activity conditions, aiming for a more accurate prediction of student performance.

\end{abstract}