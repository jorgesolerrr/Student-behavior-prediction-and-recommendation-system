\begin{opinion}
    En los últimos años, en las instituciones educacionales existe una marcada tendencia hacia el uso de los Entornos Virtuales de Aprendizaje, los cuales permiten la propagación de su instrucción a mayor número de personas, apoyándose en las facilidades que ofrecen estas plataformas.  

En estos entornos se produce una gran cantidad de información valiosa referente al aprendizaje de los alumnos que no siempre es objeto de análisis por parte de la institución, el profesorado, incluso del estudiantado, en aras de lograr mejores resultados en su proceso de enseñanza-aprendizaje.  

La Minería de Datos Educacionales (MDE), es un campo de estudio que aporta un conjunto de técnicas que facilitan el análisis de grandes repositorios de datos generados o relacionados con las actividades de aprendizaje en los centros educativos, con el objetivo de orientar mejor el proceso de instrucción, evaluar el comportamiento del desempeño de sus estudiantes, desarrollar un mejor trabajo colaborativo en los educandos, entre otras muchas acciones proveniente del análisis efectuado.  

El trabajo de investigación que se aborda en esta tesis, parte del análisis de las técnicas de minería de datos que faciliten la extracción de la información educacional y a partir de ella poder realizar predicciones del comportamiento del desempeño que llevan los estudiantes de un determinado curso, en aras de facilitar el proceso de enseñanza-aprendizaje.  

El autor primeramente ha tenido que dedicar un tiempo considerable a profundizar en la plataforma Moodle desde el punto de vista usuario y desarrollador, profundizar en los registros necesarios para obtener la información necesaria para aplicar las técnicas de aprendizaje de máquina, determinar aquellas que se fueran mejores para establecer las posibles predicciones y brindarla a la institución para su posterior análisis para la toma de decisiones.  

Ha tenido que realizar un estudio del estado del arte para determinar el camino a seguir, así como profundizar en las estructuras de la plataforma para poder insertarse en el contexto de esta. Para ello, ha tenido que consultar diferentes bibliografías, en su mayoría en idioma inglés.  

En el desarrollo de esta tesis el estudiante evidenció su motivación por la investigación, la cual ha desarrollado de manera independiente, con profesionalidad, aportando sugerencias válidas para el desarrollo de la misma. Demostró el dominio alcanzado en los contenidos estudiados y los resultados alcanzados avalan la calidad de la plataforma instrumentada.  

Es de destacar la constancia y dedicación mostrada en el desarrollo de la investigación, así como las acciones que tuvo que desarrollar para recopilar los datos de los casos de estudios presentado, que no solo se redujo a los presentados en la memoria escrita.  

Consideramos que el autor posee las habilidades necesarias de un profesional en Ciencia de la Computación, evidenciándose en la aplicación de los conocimientos adquiridos al desarrollo de otras áreas del saber, por tales razones proponemos al tribunal se le otorgue la calificación de excelente (5).
\end{opinion}